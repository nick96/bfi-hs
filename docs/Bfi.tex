\haddockmoduleheading{Bfi}
\label{module:Bfi}
\haddockbeginheader
{\haddockverb\begin{verbatim}
module Bfi (
    Construct(Decrement, Increment, Print, Input, Loop),  main
  ) where\end{verbatim}}
\haddockendheader

Brainfuck interpreter.\par
This is an interpreter for the Brainfuck language, as defined in at
\url{http://esolangs.org/wiki/brainfuck}.\par
\section*{Brainfuck language:}\begin{description}
                              \item[\haddocktt{+}] Increments the current cell.
                              \end{description}\begin{description}
                                               \item[\haddocktt{>}] Move the current cell to the right.
                                               \item[\haddocktt{<}] Move the current cell to the left.
                                               \item[\haddocktt{-}] Decrements the current cell.
                                               \item[\haddocktt{.}] Print the character indicated my the current cell value.
                                               \item[\haddocktt{,}] Input a character an store it in the current cell.
                                               \item[\haddocktt{{\char 91}}] Jump past the matching {\char 93}, if the current cell value is 0.
                                               \item[@{\char '134}] @{\char 93} If current cell value is non-zero, go back to the matching {\char 93}.
                                               \end{description}Anything other than the above characters is considered a comment.\par
                                                                The current cell value begins at zero.\par
                                                                \section*{Interpreter process:}This interpreter works in two stages. The first stage parses the the
program, breaking it up into a list of \haddockid{Construct}s. The second stage
then goes over this list, executing the appropriate functions. My
rational for this two step process, rather than just going straight to
the execution stage, is because it gives me more flexibility. I can
include optimizations without having to rewrite everything and in the
future I'll be able to turn it into a JIT compiler, or even a straight
up compiler.\par
                                                                                               
\begin{haddockdesc}
\item[\begin{tabular}{@{}l}
data\ Construct
\end{tabular}]\haddockbegindoc
\haddockbeginconstrs
\haddockdecltt{=} & \haddockdecltt{Decrement} & \\
\haddockdecltt{|} & \haddockdecltt{Increment} & \\
\haddockdecltt{|} & \haddockdecltt{Print} & \\
\haddockdecltt{|} & \haddockdecltt{Input} & \\
\haddockdecltt{|} & \haddockdecltt{Loop [Construct]} & \\
\end{tabulary}\par
Structure to represent the data constructs in a Brainfuck program.\par

\end{haddockdesc}
\begin{haddockdesc}
\item[\begin{tabular}{@{}l}
main\ ::\ IO\ ()
\end{tabular}]\haddockbegindoc
Main entry point for the program.\par

\end{haddockdesc}